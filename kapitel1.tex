\chapter{Einleitung}
\label{ch:einleitung}
Die  Konversationen  rund  um  das  Thema  Pflegerobotik  sind  in  den  letzten  Jahren  merklich  häufiger  geworden.  Dies  liegt  einerseits  daran,  dass  das  Gesundheitswesen  mit  Herausforderungen  konfrontiert  ist,  für  die  dringend  innovative  Lösungen  benötigt  werden.  Dazu  zählen  die  demografische  Entwicklung  und  der  Mangel  an  Fachkräften.  Andererseits  werden  von  Robotikherstellern  und  Universitäten  vermehrt  Projekte  vorgestellt, welche einen Nutzen für die Praxis vorweisen können. Im folgenden wird der Einsatz sozial interagierender Roboter in der Altenpflege durchleuchtet. Dabei wird in \ref{sec:values} auf die Wechselwirkung zwischen Werten und Design eingegangen. In Sektion \ref{sec:täuschung} wird eine ethische Reflektion zum Thema der Täuschung im bereich der Robotik dargelegt. Fragen der Privatheit die diesbezüglich auftauchen werden in \ref{sec:priv} behandelt. Anschließend werden Bedenken bisheriger Ansätze geäußert.





%Wie  kürzlich  in  einer  Forschungsarbeit  von  Bilyea  etal.  (2017)  ausgeführt,  haben  die  meisten   technologischen   Fortschritte   im   Bereich   der   Pflegeroboter   in   den   letzten   12Jahren  stattgefunden.  Es  ist  deshalb  wenig  überraschend,  dass  Forschung  in  diesem  Bereich  noch  in  den  Kinderschuhen  steckt.  %Frühe  Arbeiten  von  Scopelliti  etal.  (2005) zeigen,  dass  die  Akzeptanz  von  und  Einstellung  zu  Robotern,  insbesondere  bei  älteren  Menschen,  nicht  nur  von  den  praktischen  Vorteilen  abhängen,  sondern  aus  komplexen  Beziehungen  zwischen  den  kognitiven,  affektiven  und  emotionalen  Komponenten  der  Menschen  und  ihren  Vorstellungen  von  Robotern  entstehen.  Sie  fanden  heraus,  dass  ältere  Menschen  ein  höheres  Misstrauen  gegenüber  der  Technik  hatten  und  diese  auch  komplizierter  zu  bedienen  fanden.  In  ähnlichen  Studien  hat  sich  gezeigt,  dass  ältere  Menschen auch früher aufgaben, wenn sie Schwierigkeiten hatten (Giuliani etal. 2005). Andererseits  stellten  Stafford  etal.  (2014)  fest,  dass  Bewohner  von  Alters-  und  Pflege-heimen, die physische Einschränkungen hatten, einen Roboter eher benutzten als körper-lich gesunde Bewohner. Ähnlich haben Tinker und Lansley (2005) und Pain etal. (2007)dokumentiert, dass ältere Menschen besonders bereit waren, Technologie zu akzeptieren, wenn sie sich an einen spezifischen Bedarf richtete und ihnen somit mehr Unabhängig-keit brachte. Forschung, die als Kernpunkt die Interaktion zwischen Mensch und Robo-ter  hat,  war  bisher  meist  beobachtend,  unterlag  künstlichen  Bedingungen  und  wurde  nur  mit  Fotos  und  Videos  von  Robotern  durchgeführt,  statt  experimentell  und  in  rea-len  Umgebungen.  Daten  wurden  überwiegend  durch  Fragebogen  (Giuliani  etal.  2005; 
%48M. Früh und A. GasserScopelliti  etal. 2005),  Nachbearbeitungsinterviews  (Birks  etal.  2016)  oder  Fokus-gruppen  (Broadbent  etal.  2012;  Wu  und  Rigaud  2012)  gesammelt  anstelle  von  Inter-aktionen mit Robotern in der Zielu [3]%

%Bei  älteren  Menschen  ist  die  Akzeptanz  geringer  als  bei  jüngeren  Menschen.  Dennoch  zeigen  Umfragen,  dass  immer  mehr  Seniorinnen und Senioren dem Einsatz von Pflegerobotern positiv gegenüberstehen. Bei-spielsweise konnten sich bei einer repräsentativen Forsa-Umfrage, welche im April 2016 im Auftrag des BMBF durchgeführt wurde, 83 der Befragten grundsätzlich vorstellen, im Alter einen Serviceroboter in ihrem eigenen Zuhause zu nutzen, um dort länger woh-nen zu können (Forsa. Politik- und Sozialforschung GmbH 2016). [4]









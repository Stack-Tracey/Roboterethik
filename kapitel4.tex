\chapter{Fazit}
\label{ch:fazit}
Häufig wird der Kritikpunkt vorgebracht, dass Roboter keine echte Zuwendung ersetzen können. Es wurde festgestellt, dass Roboter zwar zur Stimmungsaufhellung von Demenzpatienten beitragen können, aber nicht die Wirkung von zwischenmenschlichen Kontakten erzielen \cite[79]{vier}. Es besteht die Gefahr einer Abnahme sozialer Interaktion. Pflegebedürftige könnten zunehmend vernachlässigt werden, wobei die Vernachlässigung möglicherweise damit entschuldigt wird, dass Roboter emotionale und körperliche Bedürfnisse bedienen. Es gibt aber auch Evidenz für den therapeutischen Nutzen von simulierten sozialen Stimuli. Demenzkranke Patienten könnten so vom Roboter angeregt werden, mit ihrem sozialen Umfeld zu interagieren [ebd.: 79]. Wenn ein Roboter primär eingesetzt wird, um Gesellschaft zu leisten, wäre es möglich, dass er registriert, wieviel Zeit Pflegebedürftige ohne soziale Interaktion verbracht haben \cite[38]{sharky}. Die Roboter könnten eine Warnung abgeben, wenn die Pflegebedürftigen zu lange Zeiträume allein verbracht haben [ebd.: 31]. Roboter in der Pflege können positive Effekte haben, sollten sie die Abhängigkeit vom Pflegepersonal reduzieren und eine längere Selbstständigkeit im eigenen Zuhause ermöglichen [ebd.: 31]. Der Einsatz von Robotern in der Pflege kann zu einem Zeitgewinn führen, es besteht jedoch die akute Gefahr, dass die gewonnene Zeit aus ökonomischen Gründen nicht der zu pflegenden Person zugutekommt. Vielmehr kann der Robotereinsatz dazu führen, dass eine höhere Anzahl an Patienten von der gleichen Anzahl an Pflegekräften oder dieselbe Anzahl an Patienten von einer geringeren Anzahl an Pflegekräften betreut wird \cite[143]{sparrow}. Die in Abschnitt \ref{sec:priv} dargelegten Erläuterungen könnten einen Verlust von Privatsphäre und Einschränkung der persönlichen Freiheit bedeuten. Täuschung und Infantilisierung könnten vorkommen, wenn Pflegebedürftige dazu angeregt werden mit Robotern eine gleichwertige Beziehung aufzubauen (vgl. \ref{sec:täuschung}). Weitere Richtlinien sollten durch umfassende Auswertungen der Auswirkungen eines Robotereinsatzes auf Pflegebedürftige erstellt werden.


%Im Gegensatz zu Computer Ethics allgemein, wirft der Einsatz von Robotern weitergehende Fragen auf. Roboter sind mobil und können Individuen verfolgen. Sie verkörpern in ihrem Äußeren Lebewesen. Ihre Erscheinungsform kann dazu führen, dass sie auch in Bereichen akzeptiert werden, in denen Videoüberwachung unter anderen Bedingungen nicht vorstellbar wäre (siehe \ref{sec:values}. Ihre den Lebewesen nachempfundene Erscheinungsform kann zudem bewirken, dass den Robotern soziale Fähigkeiten zugetraut werden, welche sie nicht besitzen.


%Ein Bereich, der durch Richtlinien von Sharkey und Sharkey geregelt werden könnte, ist die Zeitdauer, die Pflegebedürftige allein, nur in Begleitung eines Roboters zurückgelassen werden dürfen \cite[38]{sharky}. Es könnte eine Vorschrift geben, dass jeder Roboter erst fragen muss, ob körperlicher Kontakt akzeptiert wird, bevor ein Roboter einen Menschen stützt oder anhebt.












   


